\section{Schwierigkeiten und Herausforderungen}
Es ist klar, das ein Projekt, welches einen Angriff auf ein Krypto-Verfahren
vorstellt, nicht trivial ist. Das f�hrte w�hrend der Bearbeitungszeit zu einigen
Schwierigkeiten und Herausforderungen, die nun in diesem Kapitel vorgestellt
werden.\par\bigskip

\subsection{Unbekannte Konventionen}
Innerhalb des Papers wurden h�ufig Ausdr�cke der folgenden Art genannt:
\begin{align*}
a = (a_0,a_1,a_2,a_3)
\end{align*}
In diesem Beispiel sollte eine 32-Bit Zahl $a$ in 4 Bytes aufgesplittet werden.
Nun wurde jedoch an keiner Stelle erl�utert, ob $a_0$ das h�chstgewichtete oder
das niedrigstgewichtete Byte darstellt. Wir sind von letzterem ausgegangen, was
sich als Fehler herausstellte. Die Folge war, das zwar die Implementierung von
\emph{FEAL-4} richtig zu funktionieren schien, jedoch die Attacke aufgrund
falscher Zusammenh�nge keine L�sungen finden konnte. Erst das Betrachten des
folgenden Ausdruck gab Klarheit:
\begin{align*}
C = (C_L, C_R)
\end{align*}
Dieser Ausdruck beschreibt das Teilen einer 64-Bit Zahl $C$ in ihre zwei 32-Bit
H�lften. Die Struktur ist die selbe, wie in dem Ausdruck davor mit $a$. Wir
sehen, dass die linke, also die h�her gewichtete, H�lfte von $C$ als erstes
Element in der Klammer steht. Dies lie� uns darauf schlie�en, das $a_0$
tats�chlich das h�chstgewichtete Byte von $a$ sein muss.\par\bigskip

\subsection{Erschwerte Fehlersuche}
In einem Krypto-Verfahren werden die meisten Operationen auf Bit-Ebene
\newline durchgef�hrt. Dort k�nnen sich schnell Fehler einschleichen, die nur schwer
auffindbar sind. Und der Fakt, das die meisten Funktionen dem Zweck dienen ihren
Input zu verschl�sseln, tr�gt der Fehlersuche nicht gerade positiv
bei.\par\bigskip Abhilfe hat da der Verifizierer geleistet, der in Abschnitt
\ref{subsection:Verifizierer} vorgestellt wurde. Denn nur so konnte garantiert werden, das alle Funktionen
korrekt laufen und nicht die Fehlerquelle darstellen k�nnen. Tas�chlich wurde
der Verifizierer erst in der zweiten Iteration unserer Entwicklung hinzugef�gt,
nachdem wir die Suche nach einem Fehler nach mehreren Wochen aufgegeben hatten.
Durch das Verifizieren wurde uns aber bewusst, das der Fehler nicht von uns,
sondern dem Paper ausging.\par\bigskip

\subsection{Fehler in der Quelle}
Nach mehreren Wochen der Suche nach einem Fehler in unserer Implementierung sind
wir auf einen Fehler innerhalb der Papers von Murphy gesto�en. In einem Teil der
Attacke behauptet er:
\begin{align*}
V^{12} = V^{13}\\
V^{14} = V^{15}
\end{align*}
Die zweite Behauptung war in unseren Durchl�ufen der Attacke nie gegeben. Um den
Fehler zu finden, m�ssen wir zu erst wissen, wie $V^i$ berechnet wird:
\begin{align*}
V^i = M_1 \oplus G(P^i_L \oplus P^i_R \oplus N_1) = M_1 \oplus G(Q^i \oplus N_1)
\end{align*}
Dabei sind $M_1$ und $N_1$ Schl�sselkonstanten. Das hei�t der Zusammenhang von
$V$ Werten muss �ber die Plaintextbl�cke erfolgen. In Gleichung (4.4) im Paper
gibt Murphy beim W�hlen der Plaintexte vor:
\begin{align*}
P^{15}_R = P^{15}_L \oplus Q^{13}
\end{align*}
Das f�hrt wiederum zu folgenden Gleichungen f�r $V^{15}$:
\begin{align*}
V^{15}&=M_1\oplus G(P^{15}_L\oplus P^{15}_L\oplus Q^{13}\oplus N_1)\\
&=M_1\oplus G(Q^{13}\oplus N_1)\\
&=M_1\oplus G(P^{13}_L\oplus P^{13}_R\oplus N_1)=V^{13}
\end{align*}
Das bedeutet, wenn wir den Zusammenhang $V^{14} = V^{15}$ beabsichtigen, muss
$P^{15}_R$ folgenderma�en gebildet werden:
\begin{align*}
P^{15}_R = P^{15}_L \oplus Q^{14}
\end{align*}
Nach dieser Korrektur konnte unser Implementierung auch endlich eine
erfolgreiche Attacke durchf�hren.\par\bigskip
