%%This is a very basic article template.
%%There is just one section and two subsections.
\documentclass{article}
\usepackage[ansinew]{inputenc}
\usepackage{listings}
\usepackage{caption}
\usepackage{color}
\usepackage{xcolor}
\usepackage{cite}
\usepackage{amsmath}
\usepackage{mathtools}
 \definecolor{middlegray}{rgb}{0.5,0.5,0.5}
 \definecolor{lightgray}{rgb}{0.8,0.8,0.8}
 \definecolor{orange}{rgb}{0.8,0.3,0.3}
 \definecolor{yac}{rgb}{0.6,0.6,0.1}

\renewcommand{\lstlistingname}{Codebeispiel}

 \lstset{
   basicstyle=\scriptsize\ttfamily,
   keywordstyle=\bfseries\ttfamily\color{orange},
   stringstyle=\color{green}\ttfamily,
   commentstyle=\color{middlegray}\ttfamily,
   emph={square}, 
   emphstyle=\color{blue}\texttt,
   emph={[2]root,base},
   emphstyle={[2]\color{yac}\texttt},
   showstringspaces=false,
   flexiblecolumns=false,
   tabsize=2,
   numbers=left,
   numberstyle=\tiny,
   numberblanklines=false,
   stepnumber=1,
   numbersep=10pt,
   xleftmargin=15pt,
   captionpos=b,
   sensitive=true
 }

\lstdefinestyle{customc}{  
  belowcaptionskip=1\baselineskip,
  breaklines=true,
  frame=L,
  xleftmargin=\parindent,
  language=C,
  otherkeywords={uint32_t,uint8_t,uint64_t,uint16_t,triplet},
  showstringspaces=false,
  basicstyle=\footnotesize\ttfamily,
  keywordstyle=\bfseries\color{green!40!black},
  commentstyle=\itshape\color{purple!40!black},
  identifierstyle=\color{blue},
  stringstyle=\color{orange},
  sensitive=true
}

\lstset{style=customc, language=C}

\setlength{\parindent}{0pt}
%%=================================


\title{Implementierung der FEAL-Differentrial-Cryptanalysis Attacke nach Murphy}
\author{
	Lukas Becker\\
	\and
	Juri Golanov\\
}


\begin{document}

\maketitle{}
\newpage
\renewcommand{\contentsname}{Inhaltsverzeichnis}
\renewcommand{\listfigurename}{Abbildungsverzeichnis}
\renewcommand{\refname}{Quellen}
\tableofcontents

%%Kurze Einleitung. Was ist die Problemstellung?
%%Was wird in diesem Paper erl�utert
\section{Einleitung}
Seit jehher herrscht in der Kryptographie ein Rennen zwischen Forschern, die
neue Verfahren entwickeln und Leuten, welche die Schwachstellen in den Verfahren
suchen, um diese f�r ihre Zwecke auszunutzen. Bei der Suche nach \emph{dem}
sicheren Krypto-Verfahren helfen Paper, wie das von Sean
Murphy\cite{attackePaper}. Sie zeigen den Forschern die Schwachstellen ihrer
Algorithmen auf und wie diese in einer Attacke ausgenutzt werden. Durch diese
Erkenntnisse k�nnen dann alte Verfahren verbessert oder neue entwickelt werden,
um die jetzt bekannten Schw�chen zu beseitigen.\par\bigskip

In der folgenden Ausarbeitung werden wir uns der Implementierung der 
\emph{Krypto-Attacke auf den FEAL Algorithmus mit 20 Plaintextbl�cken oder weniger}\cite{attackePaper} von
Sean Murphy befassen. 

\subsection{Aufbau}
Zun�chst werden wir das FEAL Krypto-Verfahren an sich beleuchten. Dazu geh�ren
einmal der Aufbau der Logik, sowie die verwendeten Algorithmen und Funktionen.
\par
Im n�chsten Schritt wird auf die von Sean Murphy entwickelte Attacke
eingegangen. Hier wird vorallem aufgezeigt welche Schw�chen Murphy in dem
Verfahren entdeckt hat und wie er diese ausnutzt.
\par
Nach der Theorie folgt dann die Implementierung der Attacke. Dieses Kapitel
beschreibt �berwiegend den Projektverlauf vom ersten Auseinandersetzen mit dem
Paper bis hin zum fertigen Programm.
\par
Im Anschluss wird ein Fallbeispiel einer Attacke durchgespielt, um zu
veranschaulichen wie das Programm, also die Attacke, vorgeht, um verschl�sselte
Texte ohne Wissen des Schl�ssels zu entschl�sseln.
\par
Danach wird auf Probleme eingangen, denen wir beim Bew�ltigen des Problems
begegnet sind, sowie der resultierende L�sungsweg.
\par
Abschlie�end folgt eine kurze Konklusion zu dem fertigen Projekt.



%%Erkl�rung wie der FEAL Algorithmus funktioniert. (Bezug auf Paper!)
\section{FEAL}
FEAL-N steht f�r \emph{Fast Data Encipherment Algorithm} und ist eine
Blockchiffre, welches auf dem Feistel-Algorithmus basiert, zudem ist es ein symmetrisches 
Kryptoverfahren. FEAL wurde im Jahre 1987 von dem Entwicklerteam Akihiro Shimizu
und Shoji Miyaguchi des japanischen Telefonkonzerns \emph{Nippon Telegraph and
Telephone} (NTT) ver�ffentlicht \cite{fealPaper}. Das Ziel der Entwicklung war
es einen schnellen Verschl�sselungsalgorithmus zu schaffen, der sich effizient in
Software zu implementieren lie�. Es sollte eine Alternative zu dem symmetrischen
Verschl�sselungsalgorithmus \emph{Data Encryption Standard} (DES) darstellen,
welches von der US-Regierung entwickelt wurde und sich damals nur leicht in spezielle
Hardware implementieren lie�. \par\bigskip Das N in FEAL-N repr�sentiert die
Anzahl der Runden der Feistel-Blockchiffren-Operationen auf 64-Bit gro�en
Bl�cke bestimmt durch 64-Bit gro�e Schl�ssel. Diese Ausarbeitung befasst sich
ausschlie�lich mit der Version FEAL-4.\par\bigskip Die folgenden Punkte zeigen
auf, wann, wo und von wem die verschiedenen Versionen von FEAL erfolgreich
gebrochen werden konnten:\par

\begin{itemize}
\item \textbf{FEAL-4} noch im gleichen Jahr 1988 auf der Eurocrypt '88 von B.
den Boer
\item \textbf{FEAL-4} im Jahr 1990 von Sean Murphy mit differentieller
Kryptoanalyse unter Verwendung 20 gew�hlter Plaintextbl�cke (Thema dieser Ausarbeitung)
\item \textbf{FEAL-8} im Jahr 1989 von Biham und Shamir auf der Konferenz
SECURICOM '89
\item \textbf{FEAL-N} mit einer variablen Anzahl an Runden und \textbf{FEAL-NX}
mit 128 Bit langen Schl�ssel statt 64 Bit auf der SECURICOM '91 wieder von Biham und Shamir
\end{itemize}

\par
FEAL hat sich aufgrund zahlreicher Sicherheitsm�ngel nicht durchgesetzt und
sollte bei sicherheitskritischen Anwendungen nicht mehr verwendet werden. Es dient 
heutzutage vor allem zum Testen neuer kryptoanalytischen Angriffsmethoden.
\par\bigskip
------------------------ABSATZ

\subsection{Kurzer Exkurs: Feistel}
Eine Feistel-Chiffre besteht aus einer bestimmten Anzahl an Runden, wobei
jeweils aus dem Schl�ssel ein Rundenschl�ssel gebildet wird. Die untere
Abbildung zeigt die typische Vorgehensweise des Feistel-Algorithmus.
\par

\begin{figure}[h]
\begin{center}
\includegraphics[width=8cm]{tmp/Bilder/feistel.PNG}
\caption{Feistelchiffre}
\end{center}
\end{figure}

\par
Vor jeder Runde wird der Text in eine linke (\textit{L}) und eine rechte H�lfte
(\textit{R}) eingeteilt. Dann wird auf die rechte H�lfte eine Funktion
\textit{f} angewandt, die Teile des Schl�ssels bzw. des sogenannten
Rundenschl�ssels \textit{k} zus�tzlich als Parameter mitbekommt. Das Ergebnis
wird mit XOR ($\oplus$) mit der linken Texth�lfte verkn�pft. Das Ergebnis ist
dann die rechte Texth�lfte f�r die n�chste Runde. Die alte rechte Texth�lfte
wird die neue linke. Der Parameter \textit{i} steht f�r die Anzahl der Runden.
Wie der Feistel-Chiffre im FEAL-4 angewandt wird, wird im folgendem Abschnitt behandelt.
\par\bigskip
\subsection{Funktionsweise}
In diesem Abschnitt wird detailliert auf die Vorgehensweise des Algorithmus
eingegangen, wie das Verschl�sselungsverfahren FEAL-4 funktioniert. Es werden
Funktionen vorgestellt und erkl�rt zu welchem Zweck diese dienen. Zum groben
Ablauf wird zun�chst einmal aufgezeigt wie aus dem 64-Bit Schl�ssel die Subkeys
generiert werden, danach wie Klartexte verschl�sselt und dementsprechend wieder
entschl�sselt werden. Dieser Abschnitt richtet sich nach dem Paper von Sean
Murphy \cite{attackePaper}.
\par
Zu aller erst wird die S-Box-Funktion vorgestellt, diese ist der
Grundbestandteil der wichtigsten Funktionen im FEAL. Diese sieht folgenderma�en aus:
\begin{align*}
S_i(x,y)=Rot_2((x+y+i) Mod 256)
\end{align*}
Der Parameter $i$ ist entweder 0 oder 1, $x$ und $y$ sind 8-Bit gro�e bin�re
Zahlen im Bereich von 0 bis 255. $Rot_2$ entspricht einer Rotation nach links um zwei Bit. 
Das Ergebnis/Output der S-Box-Funktion entspricht einer 8-Bit gro�en bin�ren Zahl.
\par\bigskip
\subsubsection{Generierung der Subkeys}
Aus dem 64-Bit Key sollen zw�lf 16-Bit Subkeys entstehen, welche anschlie�end
verwendet werden, um die Klartexte zu Verschl�sseln. Damit man von den Subkeys
aus nicht so einfach auf den urspr�nglichen Schl�ssel schlie�en kann, werden die
Bits des Keys in mehreren Stufen systematisch durcheinandergebracht.
\par
Als erstes wird dazu der Key in zwei H�lften aufgeteilt. Dazu werden die 32-Bit
langen Hilfsvariablen $B$ ben�tigt, diese erstecken sich von $B_{-2}$ bis $B_6$.
Dabei wird der linke Teil des Keys ($K_L$) der Variable $B_{-1}$ der rechte Teil
des ($K_R$) Keys der Variable $B_0$ zugeordnet, die Variable $B_{-2}$ ist auf
null gesetzt.
\begin{align*}
B_{-2}=0; \qquad B_{-1}=K_L; \qquad B_0=K_R
\end{align*}
\par
Die linken und rechten H�lften von $B_1$ bis $B_6$ sind die gesuchten Subkeys,
sie werden mithilfe der oben genannten Hilfsvariablen und der $f_k$-Funktion
folgenderma�en berechnet.
\begin{align*}
B_i=f_k (B_{i-2},B_{i-1}\oplusB_{i-3})
\end{align*}
In der oberen Gleichung wird die $f_k$-Funktion verwendet, diese ist f�r das
\textit{Verw�rfeln} der Bits zust�ndig. In nachstehender Abbildung wird die
$f_k$-Funktion veranschaulicht, weiterhin sind die dazugeh�rigen Gleichungen
aufgelistet.
\begin{align*}
c= f_k (a,b)
\end{align*}

\par
\begin{figure}[h]
\begin{center}
\includegraphics[width=8cm]{tmp/Bilder/fk_funktion.PNG}
\caption{\textit{fk}-Funktion}
\end{center}
\end{figure}
---------------------Bild an falscher Stelle

\begin{gather*}
d_1= a_0 \oplus a_1\\
d_2= a_2 \oplus a_3\\
c_1= S_1 (d_1,d_2 \oplus b_0)\\
c_2= S_0 (d_2,c_1 \oplus b_1)\\
c_0= S_0 (a_0,c_1 \oplus b_2)\\
c_3= S_1 (a_3,c_2 \oplus b_3)\\
\end{gather*}
\par
Die 32-Bit gro�en Bitbl�cke $a$ und $b$ stellen den Input dar. Diese werden
anschlie�end in vier 8-Bit gro�e Teilbl�cke $a_i$ und $b_i$ unterteilt ($i$ = 0,
�, 3), welche mithilfe der S-Box-Funktion und der XOR-Operation miteinander
\textit{vermischt} werden. Daraus resultieren sich die Variablen $c_i$, welche
zusammengesetzt das 32-Bit lange Ergebnis der Funktion liefert.
\par
Zu guter Letzt werden die Hilfsvariablen $B_1$ bis $B_6$ aufgeteilt und als die
endg�ltigen Subkeys verwendet. Aus den sechs 32-Bit langen Hilfsvariablen
entstehen nun die zw�lf 16-Bit langen Subkeys $K_0$ bis $K_{11}$.
\begin{align*}
K_{2(i-1)}= B_i^{L}; \qquad K_{2i-1}= B_i^{R} 
\end{align*}
\par
Aus oberer Gleichung resultiert folgendes Ergebnis in Worten: die sechs gerade
nummerierten Subkeys $K_0$ bis $K_{10}$ bilden die linken H�lften von $B$ und
die sechs ungerade nummerierten Subkeys $K_1$ bis $K_{11}$ bilden die rechten
H�lften von $B$.
\par
Im Folgendem wird der oben aufgezeigte Vorgang vereinfacht dargestellt. Dazu
dient die untere Abbildung, die die Generierung der Subkeys visuell darstellt
und verst�ndlicher macht. Zu erw�hnen ist, dass in folgender Abbildung zwei
Schritte ausgelassen wurden. Die Erstellung der Subkeys $K_8$ bis $K_{11}$ muss
man sich dazu denken.
\par

\begin{figure}[h]
\begin{center}
\includegraphics[width=8cm]{tmp/Bilder/subkeys.PNG}
\caption{Erstellung der Subkeys vereinfacht}
\end{center}
\end{figure}
\par
-----------------Bild an falscher Stelle
\par\bigskip
\subsubsection{Verschl�sselung der Klartexte}
Das Verschl�sseln eines 64-Bit gro�en Klartextblockes $P$ erfolgt erneut durch
Aufteilung des Blockes in eine linke ($P_L$) und eine rechte ($P_R$) H�lfte.
Diese werden mit den Subkeys per XOR-Operation miteinander verkn�pft und als
Initialzustand f�r den Feistel-Chiffre genommen. Dazu werden die Variablen $L_0$
und $R_0$ verwendet und werden wie folgt berechnet.
\begin{gather*}
L_0= P_L \oplus (K_4,K_5)\\
R_0= P_L \oplus P_R \oplus (K_4,K_5 ) \oplus (K_6,K_7 )
\end{gather*}
\par
Von hier aus werden nun die vier Runden des Feistel-Algorithmus angewendet.
Daf�r wird die $f$-Funktion und die ersten vier Subkeys $K_0$ bis $K_3$ benutzt.
Die $f$-Funktion ist der $f_k$-Funktion von der Form sehr �hnlich und wird in
der unteren Abbildung veranschaulicht, die entsprechenden Gleichungen sind ebenfalls
aufgef�hrt.
\begin{align*}
c=f(a,b)
\end{align*}
\par

\begin{figure}[h]
\begin{center}
\includegraphics[width=8cm]{tmp/Bilder/f_funktion.PNG}
\caption{\textit{f}-Funktion}
\end{center}
\end{figure}
-----------------Bild an falscher Stelle
\begin{gather*}
d_1= a_0 \oplusa_1\oplus b_1\\
d_2= a_2 \oplusa_3\oplus b_2\\
c_1= S_1 (d_1,d_2)\\
c_2= S_0 (d_2,c_1)\\
c_0= S_0 (a_0,c_1)\\
c_3= S_1 (a_3,c_2)\\
\end{gather*}
\par
Die Gleichung f�r den Durchlauf der vier Feistel-Runden sieht wie folgt aus, f�r
$i$=0,1,2,3 :
\begin{gather*}
L_i= R_{i-1}\\
R_i= L_{i-1} \oplus f(R_{i-1},K_{i-1})
\end{gather*}
\par
Die daraus resultierenden Ergebnisse des Feistel-Algorithmus $L_4$ und $R_4$
werden abschlie�end mit den letzten vier Subkeys $K_8$ bis $K_{11}$ per XOR
miteinander verkn�pft und den Variablen $C_L$ und $C_R$ zugewiesen.
\par
\begin{gather*}
C_L= R_4 \oplus (K_8,K_9)\\
C_R= R_4 \oplus L_4 \oplus (K_{10},K_{11})
\end{gather*}
\par
Daraus ergibt sich zu guter Letzt der verschl�sselte Ciphertextblock $C$ mit
\begin{align*}
C=(C_L,C_R).
\end{align*}
\par
Auf die gleiche Weise k�nnen wir, wenn wir den Schl�ssel kennen, jede
verschl�sselte Nachricht dekodieren, indem die oben beschriebene Vorgehensweise
einfach in umgekehrter Reihenfolge angewendet wird. Die unteren Abbildungen
vereinfachen die oben im Detail beschriebene Vorgehensweise des Verschl�sselns
und veranschaulichen als Gegenst�ck den Vorgang des Entschl�sselns.
\par
\begin{figure}[h]
\begin{center}
\includegraphics[width=10cm]{tmp/Bilder/encode.PNG}
\caption{Encode vereinfacht}
\end{center}
\end{figure}
-----------------Bild an falscher Stelle
\par
\begin{figure}[h]
\begin{center}
\includegraphics[width=10cm]{tmp/Bilder/decode.PNG}
\caption{Decode vereinfacht}
\end{center}
\end{figure}
-----------------Bild an falscher Stelle
\par

%%Erkl�rung wie die Attacke funktionieren soll. (Bezug auf Murphy!)
\section{Attacke nach Murphy}
In diesem Kapitel wird die Attacke von Sean Murphy nach dem Paper \cite{attackePaper} in den
Fokus genommen. Es wird aufgezeigt welche Schw�chen Murphy in dem Verfahren
entdeckt hat und wie er diese ausnutzt, um das Verschl�sselungsverfahren FEAL-4
zu brechen. Murphys Vorg�nger Den Boer \cite{denBoer} verwendete 10.000
Plaintexte, um den Schl�ssel zu rekonstruieren. Im Gegensatz dazu, und hier liegt die
Besonderheit, dass sich Murphy bei der differentiellen Kryptoanalyse nur auf 20
gew�hlte Plaintextbl�cke beschr�nkte. In diesem Kapitel wird auf viele
Gleichungen verwiesen, dabei entsprechen diese der Namenkonvention aus dem Paper
von Murphy \cite{attackePaper}.
\par\bigskip

\subsection{Wahl der 20 Plaintexte}
Bevor wir auf die Kryptoanalyse von Murphy eingehen, wird zun�chst einmal
erkl�rt auf welche Art und Weise die Plaintexte �berhaupt gew�hlt werden. Die
Wahl unterliegt n�mlich einem systematischen Prinzip, welche nun vorgestellt wird.
\par
Die Wahl der 20 Plaintexte erfolgt nach bestimmten Regeln, bevor diese
aufge- zeigt werden, m�ssen als erstes einige Variablen definiert werden. Dazu
werden die linken und rechten H�lften der Plaintextbl�cke ($P^{i}$) den Variablen
$P_L^{i}$ und $P_R^{i}$ zugeordnet. Auf dieselbe Art erfolgt die Zuweisung der
Ciphertextbl�cke ($C^{i}$) mit den entsprechenden linken ($C_L^{i}$) und rechten
($C_R^{i}$) H�lften. Dabei ist zu erw�hnen, dass die Z�hlervariable $i$ sich von
0 bis 19 erstreckt, damit die Anzahl auf 20 Plaintexte zustande kommt. Nun
werden die beiden H�lften eines Plaintextes mit der XOR-Operation miteinander
verkn�pft und in eine weitere neue Variable geschrieben. Analog dazu erfolgt
dasselbe mit den Ciphertexten. Dies sieht dann wie folgt aus:

\begin{align*}
Q^{i} = P_L^{i} \oplus P_R^{i} \qquad und \qquad D^{i} = C_L^{i} \oplus C_R^{i}
\end{align*}
\par
Die Wahl der Plaintexte unterliegt folgenden Regeln:

\begin{itemize}
  \item[1.] W�hle $P^{0}$, $P^{12}$, $P^{14}$, $P^{16}$, $P^{17}$, $P^{18}$,
  $P^{19}$ zuf�llig 
  \item[2.]	W�hle $P_L^{5}$, $P_L^{6}$, $P_L^{8}$, $P_L^{9}$, $P_L^{10}$,
$P_L^{11}$, $P_L^{13}$, $P_L^{15}$ ebenfalls zuf�llig, also die linken H�lften
der jeweiligen Plaintextbl�cken 
  \item[3.] Verkn�pfe folgende Plaintextbl�cke mit Bitmasken unter Verwendung
 der XOR-Operation:
    \begin{itemize}
		\item a)\quad $P_L^{1}$ = $P_L^{0}$ $\oplus$ 80800000 \\
		\item b) \quad $P_L^{2}$ = $P_L^{0}$ $\oplus$ 00008080\\
		\item c) \quad $P_L^{3}$ = $P_L^{0}$ $\oplus$ 40400000\\
		\item d) \quad $P_L^{4}$ = $P_L^{0}$ $\oplus$ 00004040
	\end{itemize}
  \item[4.] Definiere f�r die restlichen rechten H�lften folgende Gleichungen:
  	\begin{itemize}
		\item a)\quad $P_R^{i}$ = $P_L^{i}$ $\oplus$ $Q^{0}$, \quad wobei $i$ = 1, �,
		11\\
		\item b)\quad $P_R^{13}$ = $P_L^{13}$ $\oplus$ $Q^{12}$\\
		\item c)\quad $P_R^{15}$ = $P_L^{15}$ $\oplus$ $Q^{13}$
	\end{itemize}
\end{itemize}


\par
Somit wurden 7 ganze Plaintextbl�cke und 9 Plaintext-H�lften zuf�llig gew�hlt,
daraus ergeben sich 736 zuf�llige Bits von insgesamt 1280 Bits. Nun besteht die
Aufgabe darin, die Zusammenh�nge zwischen den Bits herauszufinden.
\par
\subsection{Neuformulierung des FEAL-4 Algorithmus}
Um die Kryptoanalyse von Murphy zu verstehen muss noch vorher die Umformulierung
des FEAL-4 Algorithmus erl�utert werden. Dies stellt n�mlich das Grundger�st
dar, auf welche die Attacke baut. Dabei bezieht sich Murphy auf die Methode von
Den Boer \cite{denBoer} und definiert die $G$-Funktion, welche die lineare
Eigenschaft der $f$-Funktion darstellt. Diese wird wie folgt definiert:
\begin{align*}
c = G(a)
\end{align*}
mit
\begin{align*}
d_1 &= a_0 \oplus a_1,\\
d_2 &= a_2 \oplus a_3,\\
c_1 &= S_1(d_1, d_2)\\
c_2 &= S_0(d_2, c_1)\\
c_0 &= S_0(a_0, c_1)\\
c_3 &= S_1(a_3, c_2)
\end{align*}
und somit 
\begin{align*}
f(a, b) = G(a_0, a_1 \oplus b_1, a_2 \oplus b_2, a_3)
\end{align*}

\par
Diese Funktion ist die Kernfunktion in der Attacke und wird sehr h�ufig
verwendet.
\par
Bei dieser Umformulierung geht es haupts�chlich darum, eine neue Methode zu
finden, um Nachrichten auf einem alternativen Weg zu Verschl�sseln und zu
Entschl�sseln. Dazu werden als Alternative zum Schl�ssel sechs verschiedene
Schl�sselkonstanten definiert. Diese sehen folgenderma�en aus:

\begin{align*}
M_1 &= B_3 \oplus \theta_R(B_1)\\
N_1 &= B_3 \oplus B_4 \oplus \theta_L(B_1)\\
M_2 &= \theta_L(B_1) \oplus \theta_L(B_2)\\
N_2 &= \theta_R(B_1) \oplus \theta_R(B_2)\\
M_3 &= B_5 \oplus B_6 \oplus \theta_R(B_1)\\
N_3 &= B_5 \oplus \theta_L(B_1)
\end{align*}

\par
Zu beachten ist, dass die �u�eren Bits von $M_2$ und $N_2$ null sind. Die
Theta-Funktionen ($\theta$) zentrieren lediglich die jeweiligen 32-Bit
Block-H�lften und setzen die �u�eren Bits auf null:
\begin{gather*}
\theta_L(a_0, a_1, a_2, a_3) = (0, a_0, a_1, 0)\\
\theta_R(a_0, a_1, a_2, a_3) = (0, a_2, a_3, 0)
\end{gather*}
\begin{gather*}
\theta_L(B_i) = (0, K_{2(i-1)}, 0)\\
\theta_R(B_i) = (0, K_{2i-1}, 0)
\end{gather*}
Die $\theta_L$-Funktion stellt zentriert die geraden Subkeys dar. Die
$\theta_R$-Funktion stellt zentriert die ungeraden Subkeys dar.
\par
Durch diese Schl�sselkonstanten ist es nun m�glich den FEAL-4 Algorithmus wie
folgt umzuschreiben und damit Klartexte zu verschl�sseln:

\begin{align*}
X_0 &= P_L \oplus M_1 = L_0 \oplus \theta_R(B_1),\\
Y_0 &= P_L \oplus P_R \oplus N_1 = R_0 \oplus \theta_L(B_1) = L_1 \oplus
\theta_L(B_1),\\
X_1 &= X_0 \oplus G(Y_0) = R_1 \oplus \theta_R(B_1) = L_2 \oplus
\theta_R(B_1),\\
Y_1 &= Y_0 \oplus G(X_1) = R_2 \oplus \theta_L(B_1) = L_3 \oplus
\theta_L(B_1),\\
X_2 &= X_1 \oplus G(Y_1 \oplus M_2) = R_3 \oplus \theta_R(B_1) = L_4 \oplus
\theta_R(B_1),\\
Y_2 &= Y_1 \oplus G(X_2 \oplus N_2) = R_4 \oplus \theta_L(B_1),\\
C_L &= Y_2 \oplus N_3,\\
C_R &= X_2 \oplus M_3 \oplus C_L.
\end{align*}

\par
Wendet man die oberen Gleichungen in umgekehrter Reihenfolge an, ist es m�glich
den verschl�sselten Text wieder zu entschl�sseln. Aus den sechs
Schl�- sselkonstanten ergeben sich insgesamt 160 unbekannte Bits. Wenn es uns
gelingt diese unbekannten Bits auszurechnen, erweist sich uns die M�glichkeit
jeden Ciphertext zu entziffern und ebenfalls auf den Schl�ssel
zur�ckzuschlie�en. Damit kommen wir nun im n�chsten Abschnitt zur eigentlichen
Attacke und Kryptoanalyse von Murphy.
\par\bigskip
\subsection{Kryptoanalyse des FEAL-4 Algorithmus}
In diesem Abschnitt wird die Vorgehensweise der Attacke von Murphy beschrie-
ben, um den FEAL-4 Algorithmus zu brechen. Das Ziel der Attacke ist es die sechs
Schl�sselkonstanten $M_1$, $N_1$, $M_2$, $N_2$, $M_3$ und $N_3$ herauszufinden. 
\par
Zu aller erst werden die 20 Plaintexte gew�hlt und verschl�sselt, wie im
vorherigen Abschnitt beschrieben. Damit entstehen die Plain- und
Ciphertextbl�cke $P^{i}$ und $C^{i}$, sowie $Q^{i}$ und $D^{i}$. Diese sind notwendig f�r
das weitere Verfahren.
\par
Um die dazugeh�rigen Bits in Zusammenhang zu bringen formuliert Murphy aus (2.7)
neue Gleichungen, und zwar (5.1) und die daraus folgende Gleichung (5.2). Hier
wurde von $Y_1$ ausgegangen und die Werte nach und nach mathematisch eingesetzt.
So entsteht aus den Gleichungen 
\begin{align*}
Y_1 = Y_0 \oplus G(X_1), X_1 = X_0 \oplus G(Y_0) und X_0 = P_L \oplus M_1
\end{align*}
Die folgende
\begin{align*}
Y_1 = Y_0 \oplus G(X_1) = Y_0 \oplus G(X_0 \oplus G(Y_0)) = Y_0 \oplus G(P_L
\oplus M_1 \oplus G(Y_0)) \ldots
\end{align*}
\par
Da diese Gleichungen (5.1) und die Gleichung (5.2) sehr umfangreich sind werden
sie an dieser Stelle nicht aufgelistet, sowie die anderen Gleichungen aus dem
Paper von Murphy. Um genauere Details zu erhalten, empfiehlt es sich im Paper
nachzuschlagen \cite{attackePaper}. Es soll lediglich das Prinzip verstanden
werden, dass Murphy die Werte aus (2.7) und (2.6) in Zusammenhang mit den Ciphertextbl�cken bringt.
Aus der Gleichung (5.2) schlie�t er die Gleichungen f�r die Werte $U^{i}$, $V^{i}$
und $W$.
Diese bezeichnet er als Triple und sind essenziell zur weiteren Vorgehensweise.

\begin{align*}
U^{i} &= Y_0^{i} \oplus N_3\\
V^{i} &= M_1 \oplus G(Y_0^{i})\\
W &= M_3 \oplus N_2\\
\end{align*}

\par
Von hier aus werden Schritt f�r Schritt diese Werte ausgerechnet. $W$ setzt sich
aus den Schl�sselkonstanten $M_3$ und $N_2$ zusammen und ist daher konstant.
Deshalb wird zu Beginn nach m�glichen $W$ Werten gesucht. Aus der dritten Regel
zur Wahl der Plaintexte (XOR mit Bitmasken) schlie�t Murphy auf die Gleichung (5.8)
mithilfe der linken H�lften der Ciphertextbl�cke $C_L^{0}$, $C_L^{1}$, $C_L^{2}$ und
der $D$-Werte $D^{0}$, $D^{1}$ und $D^{2}$. Diese hat folgende Form:
\begin{gather*}
G(x \oplus a) \oplus G(x \oplus b) = d\\
G(x \oplus a) \oplus G(x \oplus c) = e\\
\end{gather*}
Dies entspricht der Gleichung (3.7) im Paper \cite{attackePaper} und erfordert
$2^{17}$ Berechnungen und wird zur Berechnung gleichzeitig ausgef�hrt. Das
Ergebnis der Berechnungen aus Gleichung (5.8) liefert eine Menge an m�glichen
L�sungen f�r $W$. Um die Ergebnismenge zu reduzieren wird gepr�ft, ob die
gefundenen $W$ Werte die Gleichungen (5.9) erf�llen. Diese ist der Gleichung
(5.8) sehr �hnlich, hier werden blo� andere Ciphertextbl�cke, andere $D$-Werte
und andere Bitmasken verwendet. $W$ Werte, die die Gleichungen (5.9) nicht
l�sen, werden aussortiert. Auf diese Weise wird die resultierende Ergebnismenge
auf bis zu zehn L�sungen f�r $W$ verkleinert.
\par
Als n�chstes wird f�r jede gefundene L�sung f�r $W$ und der Gleichungen (5.10)
nun L�sungen f�r $V^{0}$ gesucht. Diese Gleichung �hnelt den beiden vorherigen,
hier werden andere $C$ und $D$-Werte benutzt und die Verwendung von Bitmasken
entf�llt. Gleichung (5.5) liefert m�gliche $U^{0}$ Werte und damit h�tten
mehrere L�sungen f�r das Triplet ($W$, $V^{0}$, $U^{0}$). Durch die Gleichung
(5.5) folgt, dass die ersten zw�lf Werte $Y_0$ = $Q^{i} \oplus N_1$ und
$G(Y_0)$ konstant sind, also f�r $i$ = 0, �, 11, folgt daraus, dass $U^{i}$ =
$U^{0}$ und $V^{i}$ = $V^{0}$. Dies stellt die Bedingung dar, um weniger als 20 L�sungen
f�r das Triplet ($W$, $V^{0}$, $U^{0}$) zu liefern.
\par
Murphy f�hrt weiter fort einzelne $U$- und $V$-Werte mit aufgestellten
Gleichungen zu suchen bis er die Werte $U^{12}$, $U^{13}$, $U^{14}$, $U^{15}$ sowie $V^{12}$,
$V^{13}$, $V^{14}$, $V^{15}$ findet. Damit ist es ihm m�glich mit der Gleichung
(5.13) m�gliche L�sungen f�r die erste Schl�sselkonstante $N_1$ auszurechnen.
Mithilfe dieser L�sungen l�sst sich aus den folgenden Gleichungen auf $M_1$ und
$N_3$ schlie�en:
\begin{align*}
Y_0 &= Q^{0} \oplus N_1\\
M_1 &= V_0 \oplus G(Y_0)\\
N_3 &= Y_0 \oplus U^{0}
\end{align*}
\par
Anschlie�end werden f�r die $P^{0}$, $P^{17}$ und $P^{18}$ die $X_1$. und
$Y_1$-Werte berechnet, um durch die Gleichungen (5.15) L�sungen f�r $M_2$
finden. Mithilfe der Konstante $M_2$ lassen sich drei L�sungen f�r $M_3$ finden,
wobei alle drei L�sungen �bereinstimmen sollten. Diese genannte Bedingung und
das Wissen, dass die �u�eren 16 Bit null sind, l�sst auf das richtige $M_2$
schlie�en. Die letzte Konstante $N_2$ errechnet sich aus $W$ und $M_3$, zu
beachten ist, dass die �u�eren 16 Bit ebenfalls null sind. Durch die Berechnung
aller Schl�sselkonstanten ist es nun m�glich nach der neuformulierten Methode 
alle Klartextbl�cke zu Verschl�sseln und alle Ciphertextbl�cke zu Entschl�sseln.
\par
Mit den letzten beiden Gleichungen im Paper, also Gleichung (5.16) und (5.17)
besteht die M�glichkeit den Schl�ssel zu rekonstruieren. Aufgrund des
Zusammenhangs zwischen $M_1$, $N_1$, $M_3$, $N_3$ und den �u�eren 16 Bits von
$B_3$, $B_4$, $B_5$, $B_6$ lassen sich die In- und Outputs der $fk$-Funktion
errechnen.
\par

%%Erkl�rung wie wir FEAL & Attacke implementiert haben.
\section{Implementierung}
Im folgenden Abschnitt werden wir sowohl auf die Implementierung des FEAL-4
Verfahren, als auch auf die der Attacke eingehen. Als Programmiersprache
wurde C gew�hlt, da ein kompiliertes Programm eine h�here Performanz
besitzt als zum Beispiel ein Java Programm, welches �ber einen Interpreter
l�uft.\par
Die Software wurde in drei Teilkomponenten unterteilt. Einmal die
FEAL Komponente, welche alle Funktionen besitzt, um die Funktionalit�t des
FEAL Verfahrens bereitzustellen. Die n�chste Komponente widmet sich ganz
der Attacke, welche von Murphy beschrieben wurde. Abschlie�end gibt es noch
eine Verifizierer Komponente. Diese dient zur Verfikation aller Funktionen,
die in den beiden anderen Komponenten erstellt wurden.\par
Bevor wir jedoch die einzelnen Komponenten uns im Detail betrachten, werden erst
einige Grundkonventionen festgelegt. Diese dienen zur Vereinheitlichung und zum 
besseren Nachvollziehen des Codes im Bezug auf das vorgegebene 
Paper\cite{attackePaper}.\par\bigskip

\subsection{Konventionen}
Da Murphy in seinem Paper die meisten Funktionalit�ten sowohl des FEAL, als
auch des Attacke-Algorithmus als mathematische Funktionen dargestellt hat, ist
der �bergang von der Theorie in die Implementierung verh�ltnisweise einfach. Um
den Bezug auf die mathematischen Funktionen nicht zu verlieren, wurde die
Mehrzahl der Variablennamen eins zu eins �bertragen. Ausnahmen waren zum
Beispiel die Bytevariablen eines gesplitteten Doppelwortes. Nehmen wir an eine
32 bit Zahl h�tte den Variablennamen \textit{a}. So haben in dem Paper die 4
Teilbytes von \textit{a} einen zus�tzlichen fortlaufenden Index, also
\textit{a0, a1, a2, a3}. Unsere Implementierung realisiert eine gesplittete 32
bit Zahl als Byte-Array der L�nge 4 und beh�lt dabei den Variablennamen aus dem
Paper. Damit �hnelt ein Zugriff auf den entsprechenden Index im Array (z.B.
\textit{a[1]}) dem aus dem Paper (\textit{a1}). Damit sich der Namensbereich des
Arrays und der initialen 32 bit Zahl nicht �berschneidet, wird der initialen
Zahl ihre Gr��enbezeichnung an den Variablennamen des Papers angehangen. In
unserem Beispiel h�tte die \textit{UInt32} Repr�sentation von \textit{a} also
den Variablennamen \textit{aDWord}.\par
F�r erstellte Funktionen gelten die selben Namenskonventionen. Falls eine
Funktion von Murphy explizit in einer mathematischen Repr�sentierung vorhanden
ist, wird der Name dieser Funktion �bernommen. Beinhaltet der Funktionname einen 
griechischen Buchstaben (z.B. $\theta$), so wird in der Implementierung der
Buchstabe anhand des repr�sentativen Wortes aus lateinischen Buchstaben
ausgeschrieben (z.B. \textit{theta}). Wird eine Funktion nicht explizit im Paper
genannt oder niedergeschrieben, so ist f�r sie ein ad�quater Funktionsname, der
die �blichen Programmierkonventionen einh�lt zu w�hlen.\par\bigskip

\subsection{Implementierung explizit formulierter Funktionen}
Wie bereits erw�hnt werden die meisten Funktionen in dem Paper von Murphy
explizit ausformuliert. Um besser nachvollziehen zu k�nnen, wie der
Implementierungsprozess einer solchen Funktion abl�uft, wird
nun die Implementierung der Funktion \textit{f} aus dem
Paper\cite{attackePaper} durchgef�hrt.\par
Wir betrachten dabei zuerst folgenden Auszug, welcher die Definition von
\textit{f} beinhaltet:\par\bigskip

Now suppose that $a_i,c_i~\epsilon~Z^8_2$ for $i=$ 0,1,2,3, and also that
$b_1,b_2~\epsilon~Z^8_2$, with $b=(b_1,b_2)~\epsilon~Z^{16}_2$ and
$a=(a_0,a_1,a_2,a_3),c~\epsilon~Z^{32}_2$ etc., then we can define
\begin{gather*}
c = f(a,b)
\end{gather*}
as follows: 
\begin{gather*}
d_1 = a_0 \oplus a_1 \oplus b_1\\
d_2 = a_2 \oplus a_3 \oplus b_2\\
c_1 = S_1(d_1,d_2)\\
c_2 = S_0(d_2,c_1)\\
c_0 = S_0(a_0,c_1)\\
c_3 = S_1(a_3,c_2)
\end{gather*}
Anhand dieser Definition l�sst sich auf alle Eigenschaften unserer zu
implementierenden Funktion schlie�en. Wir sehen, das $f$ die Parameter $a$ und
$b$ besitzt, wobei $a$ definiert ist als Konkatenation von 4 Bytes und $b$ als
eine Konkatenation von 2 Bytes. Des Weiteren k�nnen wir erkennen, das
$f(a,b)=c$, wobei $c$ als 32 bit Zahl definiert ist. Durch diese Informationen
l�sst sich auf folgende Deklaration in C schlie�en:\par\bigskip
\begin{lstlisting}[caption={Deklartation der Funktion $f$ in C}]
 /**
 * Implementierung der f Funktion aus dem Paper
 *
 * c = f(a, b)
 *
 * @param aDWord - a
 * @param b		 - b
 *
 * @result c (32 bit)
 */
uint32_t f(uint32_t aDWord, uint16_t b);
\end{lstlisting}
Wir k�nnen in der Definition erkennen, das $a$ und $b$ nicht im Ganzen, sondern
ihre jeweiligen Bytes verwendet werden. Das hei�t, das bevor wir die Operationen
aus der Definition implementieren, m�ssen wir eine Funktion aufrufen, die uns
$a$ und $b$ in Bytes aufsplittet. Zudem m�ssen am Schluss die 4 Bytes
$c0,c1,c2,c3$ zu dem Doppelwort $c$ zusammengef�gt werden. Dies f�hrt dann zu
folgender Implementierung von $f$:
\begin{lstlisting}[caption={Implementierung der Funktion $f$ in C},
label=lst:fFunktion]
uint32_t f(uint32_t aDWord, uint16_t b)
{
	uint8_t b2 = (uint8_t) b;
	uint8_t b1 = (uint8_t)(b >> 8);
	uint8_t a[4] = {0};
	uint8_t c0, c1, c2, c3, d1, d2;

	// Split a to a0, a1, a2, a3
	splitToBytes(aDWord, a);

	d1 = a[0] ^ a[1] ^ b1;
	d2 = a[2] ^ a[3] ^ b2;
	c1 = S(d1, d2, ONE);
	c2 = S(d2, c1, ZERO);
	c0 = S(a[0], c1, ZERO);
	c3 = S(a[3], c2, ONE);

	return bytesToUint32(c0, c1, c2, c3);
}
\end{lstlisting}
Anhand dieser Vorgehensweise wurden alle weiteren explizit ausformulierten
Funktionen implementiert. Vorallem die Implementierung des FEAL Verfahren wurde
durch diese Vorgehensweise sehr vereinfacht. Interessant ist nun zu betrachten,
wie die Attacke implementiert wurde.\par\bigskip

\subsection{Implementierung der Attacke}
\label{subsection:ImplementierungAttacke}
In \emph{Krypto-Attacke auf den FEAL Algorithmus mit 20 Plaintextbl�cken oder
weniger}\cite{attackePaper} wird die Attacke haupts�chlich anhand von Prosa
geschildert, mit zus�tzlichem Bezug auf vorher aufgestellte Gleichungen. Dabei
handelt es sich um zwei verschiedene Formen von Gleichungen, die
unterschiedliche Arten der Implementierung mit sich ziehen.\par
Die erste Form sind Gleichungen, wo ein $x$ gesucht wird, welches die Gleichung
l�st. Die Vorgehensweise bei der Suche nach $x$ ist dabei h�ufig das Pr�fen von
anderen Gleichungen, welche mit Bestandteilen von $x$ zusammenh�ngen. Wenn diese
Gleichungen alle erf�llt sind, haben wir eine L�sung f�r $x$. Dies f�hrt dazu,
das nicht nur eine, sondern mehrere L�sungen f�r $x$ gefunden werden.\par
Betrachten wir als Beispiel die Implementierung der Gleichungen (3.7) aus dem
Paper\cite{attackePaper}:
\begin{align*}
G(x \oplus a) \oplus G(x \oplus b) = d\\
G(x \oplus a) \oplus G(x \oplus c) = e
\end{align*}
Wir werden uns an dieser Stelle nur die Implementierung betrachten, welche
verdeutlicht, wie L�sungen f�r ein $x$ gefunden werden. Die Theorie zu der
Implementierung finden Sie in (3.6) des Papers\cite{attackePaper}.
\newpage
\begin{lstlisting}[caption={Implementierung zur Suche von $x$ in
(3.7)\cite{attackePaper}}] 
int getSolutionsForXFrom3_7(uint32_t aDWord, uint32_t
bDWord, uint32_t cDWord, uint32_t dDWord, uint32_t eDWord, 
uint32_t ** solutions) {

	int solutionCount = 0;	// Anzahl an Loesungen fuer x
	uint8_t a[4] = {0};
	uint8_t b[4] = {0};
	uint8_t c[4] = {0};
	uint8_t d[4] = {0};
	uint8_t e[4] = {0};

	// Allokiere Plaetze, um die Loesungen fuer x zu speichern.
	// Nehmen wir an, das 100% aller z1, z2 die Gleichung (3.2) erfuellen.
	// Dann allokieren wir 2^17 * 32 = 4194304 bit = 524288 Byte = 512 KB im Heap.
	// Damit sollten alle moeglichen Loesungen fuer x in diesem Array
	// gespeichert werden koennen.

	//2^17 hat nicht funktioniert, also 131072 ausgeschrieben...
	uint32_t *tmpPointer = malloc(131072 * sizeof(uint32_t));	


	// Split a to a0, a1, a2, a3 (analog fuer b, c, d, e)
	splitToBytes(aDWord, a);
	splitToBytes(bDWord, b);
	splitToBytes(cDWord, c);
	splitToBytes(dDWord, d);
	splitToBytes(eDWord, e);

	uint8_t z1 = 0;
	uint8_t z2 = 0;
	// Check fuer jedes z1, z2...
	for(int i = 0; i < 256; ++i)
	{
		z1 = i;
		for(int j = 0; j < 256; ++j)
		{
			z2 = j;
			// Wir checken fuer beide Gleichungen in (3.7) gleichzeitig!!!
			uint8_t alpha1 = S(z1 ^ a[0] ^ a[1], z2 ^ a[2] ^ a[3], ONE);
			uint8_t beta1  = S(z1 ^ b[0] ^ b[1], z2 ^ b[2] ^ b[3], ONE);
			uint8_t gamma1 = S(z1 ^ c[0] ^ c[1], z2 ^ c[2] ^ c[3], ONE);

			if(((alpha1 ^ beta1) != d[1]) || ((alpha1 ^ gamma1) != e[1]))
				continue;

			uint8_t alpha2 = S(alpha1, z2 ^ a[2] ^ a[3], ZERO);
			uint8_t beta2  = S(beta1, z2 ^ b[2] ^ b[3], ZERO);
			uint8_t gamma2 = S(gamma1, z2 ^ c[2] ^ c[3], ZERO);

			if(((alpha2 ^ beta2) != d[2]) || ((alpha2 ^ gamma2) != e[2]))
				continue;

			for(int k = 0; k < 256; ++k)
			{
				uint8_t x0 = k;
				for(int l = 0; l < 256; ++l)
				{
					uint8_t x3 = l;
					uint8_t s0Alpha1 = S(alpha1, x0 ^ a[0], ZERO);
					if((s0Alpha1 ^ S(beta1, x0 ^ b[0], ZERO)) != d[0])
						continue;

					if((s0Alpha1 ^ S(gamma1, x0 ^ c[0], ZERO)) != e[0])
						continue;

					uint8_t s1Alpha2 = S(alpha2, x3 ^ a[3], ONE);
					if((s1Alpha2 ^ S(beta2, x3 ^ b[3], ONE)) != d[3])
						continue;

					if((s1Alpha2 ^ S(gamma2, x3 ^ c[3], ONE)) != e[3])
						continue;

					// Jede Gleichung fuer z1, z2, x0, x3 ist korrekt.
					// Errechne x1, x2 (3.4) und speichere die Loesung fuer x ab.
					uint32_t x = bytesToUint32(x0, z1 ^ x0, z2 ^ x3, x3);
					tmpPointer[solutionCount] = x;
					++solutionCount;
				}
			}
		}
	}
	*solutions = realloc(tmpPointer, solutionCount * sizeof(uint32_t));
	return solutionCount;
}
\end{lstlisting}
Wie ab Zeile 32 zu sehen ist, traversieren wir durch mehrere for-Schleifen,
wobei jede den Wert einer Komponente �ndert, welche mit $x$ zusammenh�ngt. F�r
diese Werte wird dann gepr�ft, ob sie die n�tigen Gleichungen erf�llen (z.B.
Zeile 43). Falls nicht, kann der Wert f�r diese Komponente verworfen werden. Ist
die Gleichung erf�llt, kann weiter verfahren werden. Sollten alle gew�hlten
Komponentenwerte ihre jeweiligen Gleichungen erf�llen, kann daraus eine L�sung
f�r $x$ generiert werden (Zeile 75). Bei derartigen Funktionen werden die
verschiedenen L�sungen f�r $x$ immer in einer Pointerstruktur gespeichert und
die Anzahl der gefunden L�sungen als Return-Wert zur�ck gegeben.\par\bigskip

Die zweite Form von Gleichungen sind sind quasi Assertions. Im Laufe der Attacke
sollen an bestimmten Punkten gepr�ft werden, ob die bisher gesammelten Werte
bestimmte Gleichungen erf�llen. Dies dient in erster Linie der Reduktion
m�glicher L�sungen. Als Beispiel betrachten wir die Assertion der Gleichung
(5.5) aus dem Paper\cite{attackePaper}:\newpage
\begin{lstlisting}[caption={Implementierung der Assertion f�r Gleichung
(5.5)\cite{attackePaper}}]
/**
 * Prueft, ob die Parameter Gleichung 5.5 erfuellen:
 *
 * 	CiL ^ U0 ^ G(PiL ^ V0) ^ G(Di ^ W) = 0		i = 0 		(5.5)
 *
 * @param CiL
 * @param trippel
 * @param PiL
 * @param Di
 *
 * @return 1, wenn erfuellt ; 0, wenn nicht
 */
int doesSatisfy5_5(uint32_t CiL, struct triplet trippel, uint32_t PiL,
 uint32_t Di)
{
	if(( CiL ^ trippel.U0 ^ G(PiL ^ trippel.V0) ^ G(Di ^ trippel.W)) == 0)
		return 1;
	return 0;
}
\end{lstlisting}
Wie zu sehen, handelt sich dabei um eine einfache if-Abfrage, welche die
Gleichung repr�sentiert. In Zeile 13 wird zum ersten mal die neue Datenstruktur
\emph{triplet} genannt. Diese ist ein f�r die Attacke entwickelte Struktur,
welche auf den Gleichungen (5.3)\cite{attackePaper} beruht:\par\bigskip
\begin{lstlisting}[caption={Datenstruktur f�r die Werte aus
(5.3)\cite{attackePaper}}]
/*
 * Struct fuer das tripel, welches in 5.3 vorgestellt wird
 */
struct triplet{
	uint32_t W;
	uint32_t V0;
	uint32_t U0;
};
\end{lstlisting}
Vorteil dieser Datenstruktur ist eine konsistentere Speicherung von
zusammenh�ngenden $W, V^0$ und $U^0$ Werten. Zus�tzlich erleichtert es die
Nachvollziehbarkeit innerhalb des Codes.\par\bigskip\newpage
Ein weiterer wichtiger Teil der Attacke ist die Wahl der Plaintexte. Schwierig
war dabei 64 bit Pseudo Zufallszahlen zu generieren, denn die C eigene
Zufallszahlenfunktion \emph{rand()} liefert nur Zahlen im Bereich von 0 bis
\emph{RAND\_MAX}, welches mindestens 32767 ist. Um nun eine 64 bit Pseudo
Zufallszahl zu generieren wurde die \emph{rand()} Funktion vier mal aufgerufen
und die resultierenden Werte durch bit-shift und xor Operationen zu einer 64 bit
Zahl zusammen gef�gt. Der folgende Ausschnitt ist ein Beispiel f�r eine solche
Generierung:\par\bigskip
\begin{lstlisting}[caption={Generierung einer 64 bit Pseudo Zufallszahl}]
	P[0]   = ((uint64_t)rand() << 48)^ ((uint64_t)rand() << 32) ^
			((uint32_t)rand() << 16) ^ ((uint32_t)rand());
\end{lstlisting}
Das Herzst�ck unserer Attacke ist die \emph{attack()} Funktion. Sie beinhaltet
einen gesamten Durchlauf einer Attacke, vom W�hlen der Plaintexte bis hin zum
Berechnen der Schl�sselkonstanten. Die Attacke wurde bewusst nicht zu fein
aufgesplittet, um den Weg, der in dem Paper\cite{attackePaper} beschrieben ist,
noch nachvollziehen zu k�nnen. Wir werden an dieser Stelle nicht auf die
detaillierte Implementierung der \emph{attack()} Funktion eingehen. In unserem
Fallbeispiel werden wir die komplette Funktion durchlaufen und an wichtigen
Stellen Codeausschnitte liefern, welche in Summe eine ausreichende Erl�uterung
zur Implementierung sein sollten. Doch bevor wir das Fallbeispiel
betrachten, m�ssen wir zun�chst noch auf die dritte Komponente der
Software eingehen, den Verifizierer.\par\bigskip

\subsection{Implementierung des Verifizierers}
Der Verifizierer stellt in unserer Software eine Kontrollinstanz dar. Um
sicher zu stellen, dass jede Funktion, die f�r \emph{FEAL} oder die Attacke
geschrieben wurde korrekt funktioniert, wurde im Verifizierer jeweils eine
Testfunktion hinterlegt. Jede Testfunktion pr�ft, ob die zu pr�fende Funktion
richtig agiert. Es kann auf verschiedene Arten gepr�ft werden.\par Bei den
Funktionen, die einen Wert zur�cklieferen sollen, wird vorher ein erwartetes
Ergebnis gespeichert. Dieses wird dann mit dem Ergebnis, welches die zu pr�fende
Funktion zur�ck gibt verglichen. Nur wenn erwartetes und tats�chliches Ergebnis
gleich sind, gilt die Funktion als verifiziert.\par Betrachten wir uns als
Beispiel die Funktion $f$ aus dem Codebeispiel \ref{lst:fFunktion}. $f$ gibt f�r ein
bekanntes $a$ und $b$ eine 32 bit Zahl zur�ck. Das hei�t f�r unsere
Testfunktion, das wir ein $a$ und $b$ festlegen, die $f$ Funktion anhand des
Papers\cite{attackePaper} unabh�ngig von der zu testenden Implementierung
ausf�hren und dieses Ergebnis als Erwartung voraussetzen. Dann wird die zu
testende Funktion mit den Parametern $a$ und $b$ aufgerufen und dieses Ergebnis
gespeichert. Sollten nun das erwartete und tats�chliche Ergebnis gleich sein,
gilt $f$ als verifiziert. Codebeispiel \ref{lst:Verifizierungf} zeigt die
Implementierung einer solchen Testfunktion:\par\bigskip
\begin{lstlisting}[caption={Verifizierung der Funktion $f$}, label=lst:Verifizierungf] 
int verifyFunctionF(int withOutput)
{
	uint32_t a = 0x12345678;
	uint16_t b = 0xbcde;
	uint32_t expected = 0x012e78c7;
	uint32_t result = f(a,b);

	if(withOutput)
	{
		printf("Test f mit a = 0x%" PRIx32", b = 0x%"PRIx32 ". Expected: 0x%"PRIx32" Result: 0x%"PRIx32"\n",
				a, b, expected, result);
	}
	if(expected != result)
		return 0;
	return 1;
}
\end{lstlisting}
Andere Funktionen k�nnen mit Hilfe bereits verifizierter Funktionen auf ihre
Korrektheit gepr�ft werden. Nehmen wir als Beispiel die $decode()$ Funktion f�r
das \emph{FEAL}-Verfahren. Wenn die $encode()$ Funktion bereits verifiziert ist,
l�sst sich die Richtigkeit f�r $decode()$ einfach zeigen. Sollte $decode()$
Ciphertextbl�cke, die von $encode()$ verschl�sselt wurden, wieder in die
urspr�nglichen Plaintexte dekodieren k�nnen, so gilt $decode()$ als verifiziert.
Codebeispiel \ref{lst:Verifizierungdecode} zeigt die Implementierung der
Testfunktion f�r $decode()$:\par\bigskip
\begin{lstlisting}[caption={Verifizierung der Funktion $decode()$},
label=lst:Verifizierungdecode]
int verifyFunctionDecode(int withOutput)
{
	// Testschl�ssel
	uint64_t key = 0xFF00FF00FF00FF00;

	// Zuerst werden die 12 16 bit subkeys errechnet
	uint16_t *subkeys = compSubKeys(key);

	// Danach werden 20 Plaintexte nach der Definition aus dem Paper
	// erzeugt.
	uint64_t *P = choosePlainTexts();

	// Allokiere Speicher fuer die Ciphertextbloecke und wende das 
	// FEAL-Verfahren zur Verschluesselung an.
	uint64_t *C = malloc(20 * sizeof(uint64_t));
	for(int i = 0; i < 20; ++i)
	{
		C[i] = encode(P[i], subkeys);
	}

	// Entschluessel die 20 Ciphertextbloecke und
	// vergleiche, ob sie mit den urspruenglichen Plaintextbloecken 
	// uebereinstimmen.
	uint64_t decodedP[20];
	int isEqual = 1;
	for(int i = 0; i < 20; ++i)
	{
		decodedP[i] = decode(C[i], subkeys);
		if(decodedP[i] != P[i])
		{
			isEqual = 0;
		}
		if(withOutput)
		{
			printf("Urspruenglicher Plaintext: 0x%" PRIx64 "\t", P[i]);
			printf("Dekodierter Plaintext: 0x%" PRIx64"\n", decodedP[i]);
		}
	}

	return isEqual;
}
\end{lstlisting}
Auf diese Weise ist Suche nach Fehlern in der sp�teren fertigen Attacke um
einiges leichter, da man bestimmte aufgerufene Funktionen anhand ihrer
Verifizierung ausschlie�en kann.\par\bigskip Alle in diesem Kapitel genannten
Ma�nahmen f�hrten letztendlich zu einem fertigen Programm, welches erfolgreich die
Krypto-Attacke auf das FEAL-Verfahren durchf�hren kann. Im n�chsten Kapitel wird
nun ein Fallbeispiel einer solchen Attacke erl�utert.


%%Ein Beispieldurchlauf der Attacke.
\section{Fallbeispiel}
Wir werden nun ein Beispiel f�r die Attacke auf \emph{FEAL-4} mit 20
Plaintextbl�cken verfolgen. Dabei werden wir uns die gesamte Attacke �ber in der
in Abschnitt \ref{subsection:ImplementierungAttacke} vorgestellten $attack()$
Funktion befinden. Jeder wichtige Abschnitt der Attacke wird anhand eines
Code-Ausschnitts der $attack()$ Funktion erl�utert. Zus�tzlich werden noch die
bis zu diesen Zeitpunkt wichtigen gesammelten Komponenten zur
Schl�sselkonstantenerzeugung aufgelistet.\par\bigskip
Bevor die Attacke starten kann, m�ssen zun�chst alle daf�r ben�tigten
Bestandteile gew�hlt werden. Zuerst muss ein 64 bit Schl�ssel gew�hlt
werden:\par
\begin{lstlisting}[caption={Wahl des Schl�ssels}]
// Testschl�ssel
uint64_t key = 0xFF00FF00FF00FF00;
\end{lstlisting}
Danach wird aus dem Schl�ssel die 12 16 bit Subschl�ssel generiert:\par
\begin{lstlisting}[caption={Generieren der Subschl�ssel}]
// Zuerst werden die 12 16 bit subkeys errechnet
uint16_t *subkeys = compSubKeys(key);
\end{lstlisting}
Da bei der Attacke von 20 gew�hlten Plaintextbl�cken und deren entsprechenden
Ciphertextbl�cken ausgegangen wird, m�ssen die Plaintexte gew�hlt und dann mit 
Hilfe der Subschl�ssel verschl�sselt werden:\par
\begin{lstlisting}[caption={Generieren der Plain- und Ciphertextbl�cke}]
// Danach werden 20 Plaintexte nach der Definition aus dem Paper erzeugt.
uint64_t *P = choosePlainTexts();

// Allokiere Speicher fuer die Ciphertextbloecke und wende das
//  FEAL-Verfahren zur Verschluesselung an.
uint64_t *C = malloc(20 * sizeof(uint64_t));
for(int i = 0; i < 20; ++i)
{
	C[i] = encode(P[i], subkeys);
}
\end{lstlisting}
Da laut Definition der 19. und 20. Plaintextblock zuf�llig gew�hlt werden,
wurden beide so gew�hlt, dass sie konkateniert den String 'FEAL is safe!!'
repr�sentieren. Am Ende unserer Attacke versuchen wir diesen String, durch
Entschl�sseln des 19. und 20. Ciphertextblocks mittels unserer gefundenen
Schl�sselkonstanten, wieder zu erlangen.\par\bigskip
Zus�tzlich m�ssen noch die $Q$-Werte f�r jeden Plaintextblock und $D$-Werte f�r
jeden Ciphertextblock berechnet werden. Beide Werte repr�sentieren eine xor
Operation aus der linken und rechten H�lfte des entsprechenden Textblocks. Da
die linken und rechten H�lften der Textbl�cke im Laufe der Attacke noch h�ufiger
verwendet werden, werden diese auch seperat abgespeichert:\par
\begin{lstlisting}[caption={Speichern der Q- und D-Werte}]
for(int i = 0; i < 20; ++i)
{
	CL[i] = (uint32_t)(C[i]>>32);
	CR[i] = (uint32_t)(C[i]);
	PL[i] = (uint32_t)(P[i]>>32);
	PR[i] = (uint32_t)(P[i]);
	D[i]  = CL[i] ^ CR[i];
	Q[i]  = PL[i] ^ PR[i];
}
\end{lstlisting}
Nun beginnt die wahre Attacke. An dieser Stelle wird nicht detailliert auf die
entsprechenden Gleichung aus dem Paper\cite{attackePaper} eingegangen. Es werden
lediglich im Paper vorhandene Nummer f�r die entsprechende Gleichung
genannt. Dieses Fallbeispiel soll nur dazu dienen nachvollziehen zu k�nnen, wann
nach welchen Werten gesucht wird, um in der Attacke voran zu
schreiten. F�r tieferen Einblick in die Zusammenh�nge der einzelnen
Komponenten empfehlen wir das Lesen der Quelle\cite{attackePaper}.\par\bigskip
Wir beginnen die Attacke mit der Suche nach m�glichen $W$ Werten. Die Gleichung
(5.8) entspricht dem Format der Gleichung (3.7). Die entsprechende Funktion kann
L�sungen f�r diese Form von Gleichungen liefern:\par
\begin{lstlisting}[caption={Suche nach m�glichen $W$ Werten}, label=lst:FindeW]
// Finde W
uint32_t d = CL[0] ^ CL[1] ^ 0x02000000;
uint32_t e = CL[0] ^ CL[2] ^ 0x00000002;
uint32_t *wSolutions = NULL;

uint32_t wSolutionsCount = getSolutionsForXFrom3_7(D[0], D[1],
 D[2], d, e, &wSolutions);
wSolutionsCount = getSolutionsFor5_9(D[0], D[3], D[4], CL[0],
 CL[3], CL[4], &wSolutions, wSolutionsCount);
\end{lstlisting}
Der zweite Funktionsaufruf in Codebeispiel \ref{lst:FindeW},
$getSolutionsFor5\_9$, verkleinert die Anzahl an L�sungen f�r W, indem gepr�ft
wird, ob die gefundenen $W$ Werte die Gleichungen (5.9) erf�llen. Die
resultierende Ergebnismenge hat in der Regel bis zu zehn L�sungen f�r $W$. Der
folgende Output sind die Werte, die in unserem Beispieldurchlauf gefunden
wurden:\par
\begin{verbatim}
8 Solutions for W:
0x1b73b24a
0x1b73a25a
0x1b7332ca
0x1b7322da
0x9bf3b24a
0x9bf3a25a
0x9bf332ca
0x9bf322da
\end{verbatim}
Als n�chstes werden f�r jede gefundene L�sung f�r $W$ L�sungen f�r $V^0$
gesucht. Dies geschieht mit Hilfe der Gleichungen (5.10), die wiederum der Form
(3.7) entsprechen. Durch die Gleichung (5.5) f�r den ersten Plaintextblock kann
dann ein Wert f�r $U^0$ aus den gefundenen $W$ und $V^0$ Kombinationen errechnet
werden. Erf�llen diese nun die Gleichung (5.5) f�r die n�chsten zehn
Plaintextbl�cke, so haben wir eine m�gliche L�sung f�r unser $U^0,V^0,W$
Tripel:\par
\begin{lstlisting}[caption={Suche nach m�glichen Tripeln}]
// Finde V0
uint32_t *v0Solutions;
uint32_t v0SolutionCount;
struct triplet *triplets = NULL;
int tripletsCount = 0;

for(int i = 0; i < wSolutionsCount; ++i)
{
	d = CL[0] ^ CL[5] ^ G(D[0] ^ wSolutions[i]) ^ G(D[5] ^ 
		wSolutions[i]);
	e = CL[0] ^ CL[6] ^ G(D[0] ^ wSolutions[i]) ^ G(D[6] ^ 
		wSolutions[i]);
	v0SolutionCount  = getSolutionsForXFrom3_7(PL[0], PL[5], PL[6],
	 	d, e, &v0Solutions);

	triplets = realloc(triplets, (tripletsCount + v0SolutionCount) *
			sizeof(struct triplet));
	for(int j = 0; j < v0SolutionCount; ++j)
	{
		// Berechne dazugeh�riges U0 (= CL0 ^ G(PL0 ^ V0) ^ G(D0 ^ W))
		struct triplet triple = getTripletFrom5_5(CL[0], PL[0],
				v0Solutions[j], D[0], wSolutions[i]);

		// Adde nur die Tripel, die fuer die anderen Plaintexte (5.5)
		// erfuellen
		if(doesSatisfy5_5ForOtherPlaintexts(triple, PL, CL, D))
		{
			triplets[tripletsCount] = triple;
			tripletsCount++;
		}
	}
}
\end{lstlisting}
In unserem Durchlauf wurden die folgenden Tripell�sungen gefunden:\par
\begin{verbatim}
16 Solutions for Triplets:
U0: 0xd72bf37 V0: 0x4fc3d634 W: 0x1b73b24a
U0: 0xd72bf35 V0: 0x4fc356b4 W: 0x1b73b24a
U0: 0xf72bf37 V0: 0xcf43d634 W: 0x1b73b24a
U0: 0xf72bf35 V0: 0xcf4356b4 W: 0x1b73b24a
U0: 0xd72bf35 V0: 0x4fc3d634 W: 0x1b7332ca
U0: 0xd72bf37 V0: 0x4fc356b4 W: 0x1b7332ca
U0: 0xf72bf35 V0: 0xcf43d634 W: 0x1b7332ca
U0: 0xf72bf37 V0: 0xcf4356b4 W: 0x1b7332ca
\end{verbatim}\newpage
Im n�chsten Schritt versuchen wir die $U$ und $V$ Werte f�r die Plaintextbl�cke
12, 13, 14 und 15 zu finden. $U^{12}$ und $U^{14}$ lassen sich dabei einfach
anhand der Gleichungen (5.11) berechnen. Zus�tzlich zu (5.11) kann man durch (4.4)
darauf schlie�en, dass $U^{12}=U^{13}$ und $U^{14}=U^{15}$ ist. Durch die
errechneten $U$ Werte und die Gleichungen (5.12), die die Form (3.1) besitzen,
lassen sich L�sungen f�r $V^{12}$ und $V^{14}$ finden:\par
\begin{lstlisting}[caption={L�sungen f�r $V^{12}$ und $V^{14}$}]
for(int i = 0; i < tripletsCount; ++i)
{
	uint32_t U12 = triplets[i].U0 ^ Q[0] ^ Q[12];
	uint32_t U13 = U12;
	uint32_t U14 = triplets[i].U0 ^ Q[0] ^ Q[14];
	uint32_t U15 = U14;

	uint32_t *V12Solutions = NULL;
	int V12SolutionsCount = getSolutionsForXFrom3_1(PL[12], G(D[12] ^
			triplets[i].W) ^ CL[12] ^ U12, &V12Solutions);
	uint32_t *V14Solutions = NULL;
	int V14SolutionsCount = getSolutionsForXFrom3_1(PL[14], G(D[14] ^
			triplets[i].W) ^ CL[14] ^ U14, &V14Solutions);
\end{lstlisting}
Genau so wie bei $U$, sollte auch $V^{12}=V^{13}$ und $V^{14}=V^{15}$ sein. F�r
jedes $V^{12}$ und $V^{14}$ kann mit der Gleichung (5.4) gepr�ft werden, ob
diese Bedingung zutrifft. Erf�llen beide die Gleichung, so k�nnen mit den
gesammelten Werten die Schl�sselkonstanten errechnet werden:\par
\begin{lstlisting}[caption={Pr�fen, ob $V^{12},V^{14}$ (5.4) erf�llen}]
for(int j = 0; j < V12SolutionsCount; ++j)
{
	if(doesSatisfy5_4(CL[13], U13, PL[13], V12Solutions[j], D[13],
			triplets[i].W))
	{
		for(int k = 0; k < V14SolutionsCount; ++k)
		{
			if(doesSatisfy5_4(CL[15], U15, PL[15], V14Solutions[k],
					D[15], triplets[i].W))
			{
				//U, V fuer Plaintextbloecke 0-15 speichern
				uint32_t *U = malloc(20 * sizeof(uint32_t));
				uint32_t *V = malloc(20 * sizeof(uint32_t));
				for(int l = 0; l < 12; ++l)
				{
					U[l] = triplets[i].U0;
					V[l] = triplets[i].V0;
				}
				U[12] = U12; U[13] = U13; U[14] = U14; U[15] = U15;
				V[12] = V12Solutions[j];
				V[13] = V12Solutions[j];
				V[14] = V14Solutions[k];
				V[15] = V14Solutions[k];
				// Key Konstanten berechnen.
				uint32_t *calculatedKeyConstants =
						calculateKeyConstants(PL,PR,CL,CR,Q,D,U,V,
								triplets[i].W);
\end{lstlisting}
Zur Berechnung der 6 Schl�sselkonstanten gehen wir wie folgt vor. Zuerst finden
wir anhand der Gleichung (5.13) L�sungen f�r $N_1$. Wir berechnen mittels der
Struktur von Gleichung (5.13) $V^{16}$ und $U^{16}$. Wenn diese Werte die
Gleichung (5.4) erf�llen, behalten wir diese L�sung f�r $N_1$:\par
\begin{lstlisting}[caption={Finde L�sungen f�r $N_1$}]
// Finde Loesungen fuer N1
uint32_t *N1Solutions = NULL;
uint32_t N1SolutionsCount = getSolutionsForXFrom3_7(Q[0],Q[12],
	Q[14],V[0] ^ V[12], V[0] ^ V[14], &N1Solutions);

for(int i = 0; i < N1SolutionsCount; ++i)
{
	// Berechne V16. Entspricht der Struktur von (5.13)
	// nach V16 aufgeloest.
	uint32_t V16 = G(Q[0] ^ N1Solutions[i]) ^ G(Q[16] ^ 
	N1Solutions[i]) ^ V[0];
	uint32_t U16 = U[0] ^ Q[0] ^ Q[16];
	// Ueberpruefe, ob (5.4) mit V16 erfuellt ist.
	if(!doesSatisfy5_4(CL[16],U16,PL[16],V16,D[16],W))
		continue;
\end{lstlisting}
Anhand von $N_1$ lassen sich die Werte f�r $M_1$ und $N_3$  wie
folgt errechnen:\par 
\begin{lstlisting}[caption={Berechnen von $M_1$ und $N_3$}]
	// Mittels N1 lassen sich M1 und N3 errechnen.
	uint32_t y0 = Q[0] ^ N1Solutions[i];
	uint32_t m1 = V[0] ^ G(y0);
	uint32_t n3 = y0 ^ U[0];
\end{lstlisting}
Wir berechnen nun die $X_1$ und $Y_1$ Werte f�r die Plaintexte 0, 17 und 18.
Mit diesen Werten kann man durch die Gleichungen (5.15) L�sungen f�r $M_2$
finden:\par
\begin{lstlisting}[caption={Finden von L�sungen f�r $M_2$}]
		// Finde Loesungen fuer M2.
		uint32_t x1_0  = PL[0] ^ m1 ^ G(y0);
		uint32_t x1_17 = PL[17] ^ m1 ^ G(Q[17] ^ N1Solutions[i]);
		uint32_t x1_18 = PL[18] ^ m1 ^ G(Q[18] ^ N1Solutions[i]);
		uint32_t y1_0  = y0 ^ G(x1_0);
		uint32_t y1_17 = Q[17] ^ N1Solutions[i] ^ G(x1_17);
		uint32_t y1_18 = Q[18] ^ N1Solutions[i] ^ G(x1_18);

		uint32_t d = x1_0 ^ x1_17 ^ D[0] ^ D[17];
		uint32_t e = x1_0 ^ x1_18 ^ D[0] ^ D[18];

		uint32_t *M2Solutions = NULL;
		int M2SolutionsCount = getSolutionsForXFrom3_7(y1_0,y1_17,
			y1_18,d,e,&M2Solutions);
\end{lstlisting}
Als n�chstes pr�fen wir f�r jedes m�gliche $M_2$, ob die �u�eren 16 bit null
sind. Ist dies der Fall, errechnen wir mit dem $M_2$ Wert drei verschiedene
Werte f�r $M_3$ aus. Da $M_3$ eine Konstante ist, sollten alle drei Werte
�bereinstimmen:\par
\begin{lstlisting}[caption={Berechnen von $M_3$}]
for(int j = 0; j < M2SolutionsCount; ++j)
{
	// Check, ob die auesseren 16 bit 0 sind.
	if((M2Solutions[j] & 0xFF0000FF) != 0)
		continue;

	// Mit Hilfe von M2 drei Werte fuer M3 schreiben, 
	// die uebereinstimmen sollten.
	uint32_t x2_0  = x1_0 ^ G(y1_0 ^ M2Solutions[j]);
	uint32_t x2_17 = x1_17 ^ G(y1_17 ^ M2Solutions[j]);
	uint32_t x2_18 = x1_18 ^ G(y1_18 ^ M2Solutions[j]);
	uint32_t m3_0  = D[0] ^ x2_0;
	uint32_t m3_17 = D[17] ^ x2_17;
	uint32_t m3_18 = D[18] ^ x2_18;

	if((m3_0 != m3_17) || (m3_17 != m3_18))
		continue;
\end{lstlisting}
Nun fehlt nur noch die Schl�sselkonstante $N_2$. Diese l�sst sich mittels $W$
und $M_3$ berechnen, wobei die �u�eren 16 bit null sein sollten:\par
\begin{lstlisting}[caption={Berchnen von $N_2$}]
// N2 berechnen und pruefen, ob die aeusseren 16 Bit 0 sind.
uint32_t n2 = W ^ m3_0;

if((n2 & 0xFF0000FF) != 0)
	continue;
\end{lstlisting}
Da jetzt alle Schl�sselkonstanten gefunden wurden, k�nnen wir versuchen unseren
Text vom Anfang aus den letzten beiden Ciphertextbl�cken wieder zu erlangen. Der
folgende Output zeigt die gefundenen Schl�sselkonstanten im Vergleich mit den
durch Wissen des Schl�ssels errechneten, sowie der Dekodierung der letzten
beiden Ciphertextbl�cken:\par
\begin{verbatim}
Possible Key Constants:
M1: 0x5621c0cc  Berechnet: 0x5621c0cc
N1: 0xcc1ce1a   Berechnet: 0xcc1ce1a
M2: 0x40ef00    Berechnet: 0x40ef00
N2: 0x44f200    Berechnet: 0x44f200
M3: 0x1b37c0ca  Berechnet: 0x1b37c0ca
N3: 0xb227bb0   Berechnet: 0xb227bb0
FEAL is safe!!
\end{verbatim}
Im n�chsten Kapitel werden die Schwierigkeiten und Herausforderungen, die wir
w�hrend des Projekts bew�ltigen mussten, erl�utert.\par


%%Probleme bei der Bew�ltigung unserer Aufgabe
\section{Schwierigkeiten und Herausforderungen}
Es ist klar, das ein Projekt, welches einen Angriff auf ein Krypto-Verfahren
vorstellt, nicht trivial ist. Das f�hrte w�hrend der Bearbeitungszeit zu einigen
Schwierigkeiten und Herausforderungen, die nun in diesem Kapitel vorgestellt
werden.\par\bigskip

\subsection{Unbekannte Konventionen}
Innerhalb des Papers wurden h�ufig Ausdr�cke der folgenden Art genannt:
\begin{align*}
a = (a_0,a_1,a_2,a_3)
\end{align*}
In diesem Beispiel sollte eine 32-Bit Zahl $a$ in 4 Bytes aufgesplittet werden.
Nun wurde jedoch an keiner Stelle erl�utert, ob $a_0$ das h�chstgewichtete oder
das niedrigstgewichtete Byte darstellt. Wir sind von letzterem ausgegangen, was
sich als Fehler herausstellte. Die Folge war, das zwar die Implementierung von
\emph{FEAL-4} richtig zu funktionieren schien, jedoch die Attacke aufgrund
falscher Zusammenh�nge keine L�sungen finden konnte. Erst das Betrachten des
folgenden Ausdruck gab Klarheit:
\begin{align*}
C = (C_L, C_R)
\end{align*}
Dieser Ausdruck beschreibt das Teilen einer 64-Bit Zahl $C$ in ihre zwei 32-Bit
H�lften. Die Struktur ist die selbe, wie in dem Ausdruck davor mit $a$. Wir
sehen, dass die linke, also die h�her gewichtete, H�lfte von $C$ als erstes
Element in der Klammer steht. Dies lie� uns darauf schlie�en, das $a_0$
tats�chlich das h�chstgewichtete Byte von $a$ sein muss.\par\bigskip

\subsection{Erschwerte Fehlersuche}
In einem Krypto-Verfahren werden die meisten Operationen auf Bit-Ebene
\newline durchgef�hrt. Dort k�nnen sich schnell Fehler einschleichen, die nur schwer
auffindbar sind. Und der Fakt, das die meisten Funktionen dem Zweck dienen ihren
Input zu verschl�sseln, tr�gt der Fehlersuche nicht gerade positiv
bei.\par\bigskip Abhilfe hat da der Verifizierer geleistet, der in Abschnitt
\ref{subsection:Verifizierer} vorgestellt wurde. Denn nur so konnte garantiert werden, das alle Funktionen
korrekt laufen und nicht die Fehlerquelle darstellen k�nnen. Tas�chlich wurde
der Verifizierer erst in der zweiten Iteration unserer Entwicklung hinzugef�gt,
nachdem wir die Suche nach einem Fehler nach mehreren Wochen aufgegeben hatten.
Durch das Verifizieren wurde uns aber bewusst, das der Fehler nicht von uns,
sondern dem Paper ausging.\par\bigskip

\subsection{Fehler in der Quelle}
Nach mehreren Wochen der Suche nach einem Fehler in unserer Implementierung sind
wir auf einen Fehler innerhalb der Papers von Murphy gesto�en. In einem Teil der
Attacke behauptet er:
\begin{align*}
V^{12} = V^{13}\\
V^{14} = V^{15}
\end{align*}
Die zweite Behauptung war in unseren Durchl�ufen der Attacke nie gegeben. Um den
Fehler zu finden, m�ssen wir zu erst wissen, wie $V^i$ berechnet wird:
\begin{align*}
V^i = M_1 \oplus G(P^i_L \oplus P^i_R \oplus N_1) = M_1 \oplus G(Q^i \oplus N_1)
\end{align*}
Dabei sind $M_1$ und $N_1$ Schl�sselkonstanten. Das hei�t der Zusammenhang von
$V$ Werten muss �ber die Plaintextbl�cke erfolgen. In Gleichung (4.4) im Paper
gibt Murphy beim W�hlen der Plaintexte vor:
\begin{align*}
P^{15}_R = P^{15}_L \oplus Q^{13}
\end{align*}
Das f�hrt wiederum zu folgenden Gleichungen f�r $V^{15}$:
\begin{align*}
V^{15}&=M_1\oplus G(P^{15}_L\oplus P^{15}_L\oplus Q^{13}\oplus N_1)\\
&=M_1\oplus G(Q^{13}\oplus N_1)\\
&=M_1\oplus G(P^{13}_L\oplus P^{13}_R\oplus N_1)=V^{13}
\end{align*}
Das bedeutet, wenn wir den Zusammenhang $V^{14} = V^{15}$ beabsichtigen, muss
$P^{15}_R$ folgenderma�en gebildet werden:
\begin{align*}
P^{15}_R = P^{15}_L \oplus Q^{14}
\end{align*}
Nach dieser Korrektur konnte unser Implementierung auch endlich eine
erfolgreiche Attacke durchf�hren.\par\bigskip


%%Ein paar abschlie�ende Worte zu dem Projekt.
\section{Konklusion}
Die Arbeit an diesem Projekt war f�r uns sehr aufschlussreich. Eine praktische
Erfahrung zu dem bereits vorhandenen theoretischen Wissen hat unseren 
Einblick in das Thema Kryptographie noch weiter gesch�rft. Das Einarbeiten und 
Implementieren einer uns vorher noch unbekannten Attacke gab uns sehr viel
Aufschluss �ber die Herangehensweise der Entwicklung einer Attacke gegen ein
Krypto-Verfahren.\par
Zusammenfassend l�sst sich sagen, dass das Projekt ein Erfolg war und wir 
durchweg positive Erfahrungen mitnehmen.\par\bigskip
\listoffigures
\bibliographystyle{ieeetr}
\bibliography{bib}{}

\end{document}
