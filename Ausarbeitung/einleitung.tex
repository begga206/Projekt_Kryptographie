\section{Einleitung}
Seit jehher herrscht in der Kryptographie ein Rennen zwischen Forschern, die
neue Verfahren entwickeln und Leuten, welche die Schwachstellen in den Verfahren
suchen, um diese f�r ihre Zwecke auszunutzen. Bei der Suche nach \emph{dem}
sicheren Krypto-Verfahren helfen Paper, wie das von Sean
Murphy\cite{attackePaper}. Sie zeigen den Forschern die Schwachstellen ihrer
Algorithmen auf und wie diese in einer Attacke ausgenutzt werden. Durch diese
Erkenntnisse k�nnen dann alte Verfahren verbessert oder neue entwickelt werden,
um die jetzt bekannten Schw�chen zu beseitigen.\par\bigskip

In der folgenden Ausarbeitung werden wir uns der Implementierung der 
\emph{Krypto-Attacke auf den FEAL Algorithmus mit 20 Plaintextbl�cken oder weniger}\cite{attackePaper} von
Sean Murphy befassen. 

\subsection{Aufbau}
Zun�chst werden wir das FEAL Krypto-Verfahren an sich beleuchten. Dazu geh�ren
einmal der Aufbau der Logik, sowie die verwendeten Algorithmen und Funktionen.
\par
Im n�chsten Schritt wird auf die von Sean Murphy entwickelte Attacke
eingegangen. Hier wird vorallem aufgezeigt welche Schw�chen Murphy in dem
Verfahren entdeckt hat und wie er diese ausnutzt.
\par
Nach der Theorie folgt dann die Implementierung der Attacke. Dieses Kapitel
beschreibt �berwiegend den Projektverlauf vom ersten Auseinandersetzen mit dem
Paper bis hin zum fertigen Programm.
\par
Im Anschluss wird ein Fallbeispiel einer Attacke durchgespielt, um zu
veranschaulichen wie das Programm, also die Attacke, vorgeht, um verschl�sselte
Texte ohne Wissen des Schl�ssels zu entschl�sseln.
\par
Danach wird auf Probleme eingangen, denen wir beim Bew�ltigen des Problems
begegnet sind, sowie der resultierende L�sungsweg.
\par
Abschlie�end folgt eine kurze Konklusion zu dem fertigen Projekt.

